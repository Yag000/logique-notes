\subsection{Axiomes de la théorie des ensembles}

\begin{axiom} [Extensionnalité]
	$$\forall A \ \forall B ( A = B \iff \forall x (x \in A \iff x \in B))$$
\end{axiom}

\begin{axiom} [Compréhension]
	Pour toute propriété $P(x)$, exprimée dans le langage de la théorie des ensembles du premier ordre,
	$$ \forall A \ \exists B \subset A \text{ des éĺéments de } A \text{ qui vérifient } P(x)$$
	Qui est équivalent à:
	$$ \forall A \  \exists B \  \forall x (x \in B \iff x \in A \land P(x))$$
	Cet ensemble est unique par extensionnalité et il est noté $\{x \in A \mid P(x)\}$.
\end{axiom}

\begin{axiom}[Des paires]
	$$ \forall A \  \forall B \  \exists C \  \forall \ x \  (x \in C \iff x = A \lor x = B)$$
	Il est noté $\{A,B\}$.
\end{axiom}


\begin{axiom}[Réunion]
	$$ \forall A \ \exists B \ (\forall x \in B \iff (\exists Y \in A  \ x \in  Y))$$
    Noté $\bigcup A$.
\end{axiom}


