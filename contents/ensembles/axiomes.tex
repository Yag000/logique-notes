\subsection{Axiomes de la théorie des ensembles}

\begin{axiom} [Extensionnalité]
	$$\forall A \ \forall B ( A = B \iff \forall x (x \in A \iff x \in B))$$
\end{axiom}

\begin{axiom} [Compréhension]
	Pour toute propriété $P(x)$, exprimée dans le langage de la théorie des ensembles du premier ordre,
	$$ \forall A \ \exists B \subset A \text{ des éĺéments de } A \text{ qui vérifient } P(x)$$
	Qui est équivalent à:
	$$ \forall A \  \exists B \  \forall x (x \in B \iff x \in A \land P(x))$$
	Cet ensemble est unique par extensionnalité et il est noté $\{x \in A \mid P(x)\}$.
\end{axiom}

\begin{axiom}[Des paires]
	$$ \forall A \  \forall B \  \exists C \  \forall \ x \  (x \in C \iff x = A \lor x = B)$$
	Il est noté $\{A,B\}$.
\end{axiom}

\begin{axiom}[Réunion]
	$$ \forall A \ \exists B \ (\forall x \in B \iff (\exists C \in A  \land x \in  C))$$
	Noté $\bigcup A$.
\end{axiom}


\begin{axiom}[Ensemble des parties]
	$$ \forall A \exists V \forall C ( c \in B \iff \forall x ( x \in C \implies x    in A ) $$
	On note $B = \mathcal{P}(A)$.
\end{axiom}


\begin{axiom}[Infini]
	$$ \exists A (\emptyset \in A \land \forall x (x in A \implies x \cup \{x\} \in A))$$
	On peut noter cette opération $x^+$.
\end{axiom}


\begin{remarque}
	Soit $A$ un ensemble non vide, on définit alors :
	$$ \bigcap A = \left\{x \in \bigcup A \mid \forall B \ (B \in A \iff \ x \in B)\right\}$$
\end{remarque}


Soit $A$ un ensemble donné par l'axiome de l'infini, on considère l'ensemble:
$$ \left\{ B \in \mathcal{P}(A) \mid \emptyset \in B \land \forall x (x \in B \implies x^+ \in B) \right\} = \N $$
Notez bien que l'axiome de l'infini est valide pour:
$$ \N \cup \left\{ \N, \N^+, \N^{++}, \ldots \right\} $$


\begin{prop}[Couples de Wiener-Kuratowski]
	À deux ensembles $a, B$ on peut associer un ensemble, noté $(A,B)$ défini par
	$$ (A,B) = \{ \{A\}, \{A,B\} \}$$
	On a pour tout $A, B, C, D$:
	$$ \left((A, B) = (C,D) \iff (A = C \land B = D)\right)$$
\end{prop}

\begin{proof}
	Si $A = C \land B = D$, alors $(A,B) = (C,D)$.\\
	Opn suppose (A,B) = (C,D), i.e. $\{\{A\}, \{A,B\}\} = \{\{C\}, \{C,D\}\}$.\\

	\begin{itemize}
		\item Si $A \neq B$ on a $\{A\} \neq \{A,B\}$, donc $\{\{A\}, \{A,B\}\}$ n'est pas in singleton.
		      De même pour $\{\{C\}, \{C,D\}\}$.\\
		      Il y a deux (ou moins) éléments dans $\{\{C\}, \{C,D\}\}$, donc $\{C\}$ et  $\{C,D\}$ sont différents.\\
		      Et donc $C \neq D$. \\
		      Il y a un seul singleton dans $\{\{A\}, \{A,B\}\}$, donc $\{A\} = \{C\}$, donc $A = C$.\\
		      On a aussi $$\{A,B\} = \{C,D\}$$, donc $\{A, B\}\setminus \{A\} = \{C,D\}\setminus \{C\}$, donc $B = D$.
		\item Le reste est laissé en exercice pour le lecteur.
	\end{itemize}
\end{proof}


On note
$$ \Pi A = \left\{ f: A \to \bigcup A \mid \forall x \in A \ (f(x) \in x) \right\}$$



