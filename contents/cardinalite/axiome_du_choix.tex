\subsection{Axiome du choix}

On n'arrive pas à démontrer que tout ensemble infini contient un sous-ensemble dénombrable si on ne suppose pas AC.


\begin{definition}[Famille]
	$$ \left\{ (i, A_i) \mid i \in I \right\} $$

\end{definition}

\begin{axiom}[Forlumation équiavalente de l'axiome du choix I]
	Pour toute famille non vide $(A_i)_{i \in I}$ il existe une fonction
	$f : I \to \bigcup{i \in I} A_i$ qui vérifie $F(i) \in A_i, \ \forall i$
\end{axiom}

\begin{axiom}[Forlumation équiavalente de l'axiome du choix II]
	Si $(A_i)_{i \in I}$ est une famille non vide d'ensembles non vides,
	$$ \emptyset \neq \prod_{i \in I} A_i = \left\{ f: I \to \bigcup\limits_{i \in I} A_i  \mid f(i) \in A_i \right\} $$
\end{axiom}

\begin{theorem}
	Tout ensemble infini possède un sous ensemble dénombrable.
\end{theorem}

\begin{proof}
	Soit $E$ un ensemble infini $(\nexists n \in \N \mid E \cong N_{\leq n})$. Il suffit de trouver une fonction injective
	$\phi : \N \to E$. Soit $f: \Pa(E)\setminus\{\emptyset\} \to E$ du choix.\\
	On définit:
	$$ \Phi : \N \to \Pa(E) $$
	Donc $ \Phi (0) = E$.\\
	%TODO
\end{proof}




\begin{prop}
	Un ensemble est infini si et seulement si il est équipotent  une partie propre.
\end{prop}

\begin{proof}
	%TODO
\end{proof}


