\subsection{Définition par récurrence}

On peut définir $\leq$ par:
$$ x \leq y \iff \text{ il existe une suite finie } x_, S(x), S(S(x)), \ldots, S(S(\ldots S(x)) \ldots) = y $$
mais ceci n'est pas une définition du premier ordre.\\

L'objectif est de définir $+ : \mathbb{N} \times \mathbb{N} \to \mathbb{N}$:
\begin{itemize}
	\item $x + 0 = x$
	\item $x + S(y) = S(x + y)$
\end{itemize}

Mas pour cela nous avons besoin du théoreme suivant:

\begin{theorem} [Dedekind 1888]
	Si $E$ un ensemble non vide, $a \in E$ et $h : E \to E$ une fonction. Alors il existe une unique fonction $f : \mathbb{N} \to E$ telle que:
	\begin{itemize}
		\item $f(0) = a$
		\item $f(S(x)) = h(f(x))$
	\end{itemize}
\end{theorem}

\begin{proof}
	L'unicité se demontre par récurrence. Soit $g$ une fonction vérifiant les mêmes propriétés que $f$.
	\begin{itemize}
		\item $g(0) = a = f(0)$
		\item Si on a $g(x) = f(x)$ alors $f(S(x)) = h(f(x)) = h(g(x)) = g(S(x))$
	\end{itemize}
	Demontrons maintenant l'existence. Nous avons besoin d'une relation entre les éléments de $E$ et les éléments de $\mathbb{N}$, pour pouvoir
	définir une propriété de clôture. Pour cela on travaillera sur les sous ensembles de $\mathbb{N} \times E$.
	Cette propriété de clôture est inspirée par la définiton de $f$, pour les sous-ensembles $R$ de $\mathbb{N} \times E$.

	$$ Cl(A) = (0,a) \in R \ \text{et} \ \forall n \in \mathbb{N}, \forall y \in E, (n,y) \in R \implies (S(n), h(y)) \in R $$

	On peut vérifier que si $R$ est le graphe d'une fonction $f$ alors on retrouve les équations du théorème.

	L'ensemble des $R \subset \mathbb{N} \times E$ vérifiant la propriété de clôture est non vide car il contient $\mathbb{N} \times E$. On peut donc poser:
	$$ G = \bigcap\limits_{Cl(R)} R $$

	Il s'agit de montrer que $G$ est le graphe d'une fonction $f$ vérifiant les équations du théorème. Pour cela, il suffit de montrer que $G$ est une fonction.

	\begin{itemize}
		\item On a bien $Cl(G)$.
		\item Tout élément $n \in \mathbb{N}$ possède une unique image par la relation de graphe $G$. Par récurrence:
		      \begin{itemize}
			      \item Pour $n = 0$, on a $(0,a) \in G$ car $Cl(G)$.
			      \item Pour $n = S(x)$, on a $(n, h(y)) \in G$, pour $(n,y) \in G$.
		      \end{itemize}
		\item Tout élément $n \in \mathbb{N}$ possède une unique image par la relation de graphe $G$. Par récurrence:
		      \begin{itemize}
			      \item Pour $n = 0$ s'il existe $b \neq a$ tel que $(0,b) \in G$. On a que $G' = G \setminus \{(0,b)\}$ vérifie $Cl(G')$, donc
			            $G' \subset G$ ce qui est absurde.
			      \item Si $n = S(m)$. Par hypothèse de récurrence $n$ a une seule image par $G$, qu'on note $X$.On sait aussi que $S(n)$ a pour image $h(x)$.
			            Supposons que $S(n)$ ait pour image $y \neq h(x)$. On pose $G' = G \setminus \{(n,y)\}$ et on va aboutir à une contradiction en montrant que $G'$ vérifie $Cl(G')$.
			            \begin{itemize}
				            \item On note que $(0,a) \in G$ et comme $0 \neq S(n)$, on a $(0,a) \in G'$.
				            \item Soit $d \in \mathbb{N}$ et $z \in E$ tel que $(d,z) \in G'$ alors $(d,z) \in G$ et donc $(S(d), h(z)) \in G$.
				                  \begin{itemize}
					                  \item Si $d \neq m$ et $S(d) \neq S(n)$ alors $(S(d), h(z)) \in G'$.
					                  \item Si $d = m$, donc $z = x$ et alors $(S(d), h(z)) = (m, h(x)) \in G'$.
					                  \item Si $d \neq n$ alors $(S(d), h(z)) \in G'$.
				                  \end{itemize}
				                  Donc $Cl(G')$ ce qui est absurde.
			            \end{itemize}
		      \end{itemize}
	\end{itemize}
\end{proof}


\begin{exercice}
	Démontrer que chacun des axiomes de Peano sont bien nécessaire pour le théorème de Dedekind.
\end{exercice}



\begin{coro}[Définition par récurence avec paramèttre]
	Soit $A$ un ensemble et $E$ un ensemble non vide. Soit $g: A \to E$ et $h: A \times E \to E$ deux fonctions. Alors il existe une unique fonction $f: \mathbb{N} \times A \to E$ telle que:
	\begin{itemize}
		\item $f(0,y) = g(y) \ \forall y \in A$
		\item $f(S(x), y) = h(f(x,y), y) \ \forall x \in \mathbb{N}, \forall y \in A$
	\end{itemize}
\end{coro}


\begin{proof}
	On note $E^A$ l'ensemble des fonctions de $A$ dans $E$
	On veut définir par récurrence la fonction
	\begin{eqnarray} \label{eq:recurrence}
		\tilde{f} : A & \to     & E^A   \\
		x             & \mapsto & f_x : A \to E    \nonumber \\
		&         & y \mapsto f(x,y) \nonumber
	\end{eqnarray}
	\begin{itemize}
		\item Unicité: Soit $f,f'$ deux fonctions vérifiant les propriétés du corollaire. On définit $\tilde{f}$ et $\tilde{f'}$ en utilisant \ref{eq:recurrence}.
		      Donc $f = f' \iff \tilde{f} = \tilde{f'}$.
		      Maintenat on étudie juste l'unicité de $\tilde{f}$.
		      \begin{itemize}
			      \item  $ \tilde{f}(0) = g = \tilde{f'}(0)$
			      \item $ \tilde{f}(S(x)) = \tilde{h}(\tilde{f}(x))$ et $ \tilde{f'}(S(x)) = \tilde{h}(\tilde{f'}(x))$.
		      \end{itemize}
		      Comme elles vérifient les équations du théorème de Dedekind, on a $\tilde{f} = \tilde{f'}$ et donc $f = f'$.
		\item Existence: Soit \begin{eqnarray*}
			      \tilde{h}: E^A & \to & E^A \\
			      k & \mapsto & h(k(y),y)
		      \end{eqnarray*}
		      Soit $\tilde{f} : \N \to E^A$ définie par le théoreme de rñecurrence:
		      \begin{itemize}
			      \item $\tilde{f}(0) = g$
			      \item $\tilde{f}(S(x)) = \tilde{h}(\tilde{f}(x))$
		      \end{itemize}
		      Maintenant \begin{eqnarray*}
			      f : \N \times A & \to & E \\
			      (x,y) & \mapsto & \tilde{f}(x)(y)
		      \end{eqnarray*}
		      et \begin{itemize}
			      \item $f(0,y) = \tilde{f}(0)(y) = g(y)$
			      \item \begin{eqnarray*}
				            f(s(x),y) & = & \tilde{f}(S(x))(y) \\
				            & = & \tilde{h}(\tilde{f}(x))(y) \\
			            \end{eqnarray*}
		      \end{itemize}
	\end{itemize}
\end{proof}


\begin{example}
	On veut définir l'addition par récurrence comme suit:
	\begin{itemize}
		\item $x + 0 = x$
		\item $x + S(y) = S(x + y)$
	\end{itemize}
	Pour le faire on pose:
	\begin{itemize}
		\item $A = \mathbb{N}$
		\item $g = id_{\N}$
		\item $h(x,y) = S(x)$
	\end{itemize}
	Par le corollaire précédent, il existe une unique fonction $f: \N \times \N \to \N$ telle que:
	\begin{eqnarray*}
		f : \N \times \N & \to & \N \\
		f(0,y) & = & g(y) = y \\
		f(S(x),y) & = & h(f(x,y),y) = S(f(x,y))
	\end{eqnarray*}
	Ainsi, l'addition est bien définie par récurrence.
\end{example}

\begin{example}
	On veut définir une fonction $mult: \N \times \N \to \N$ telle que:
	\begin{eqnarray*}
		x \cdot 0 & = & 0 \\
		x \cdot S(y) & = & x \cdot y + y
	\end{eqnarray*}

	Comme pour la multiplication on commence par poser:
	\begin{itemize}
		\item $A = \N$
		\item $g = 0$
		\item $h(x,y) = x + y$
	\end{itemize}

	Et on a par le corollaire précédent:
	\begin{eqnarray*}
		f : \N \times \N & \to & \N \\
		f(0,y) & = & g(y) = 0 \\
		f(S(x),y) & = & h(f(x,y),y) = f(x,y) + y
	\end{eqnarray*}
	Et ainsi la multiplication est bien définie par récurrence.
\end{example}


\begin{example}[Puissances itérées de Knuth] \href{https://en.wikipedia.org/wiki/Knuth\%27s_up-arrow_notation}{Wikipedia}
	\begin{itemize}
		\item $x \uparrow 0 = 1$
		\item $x \uparrow S(y) = x^{x \uparrow y}$
		\item $x \uparrow \uparrow y = \underbrace{x \uparrow x \uparrow \ldots \uparrow x}_{y \text{ fois}}$
	\end{itemize}
\end{example}
