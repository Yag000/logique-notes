\documentclass{article}
\usepackage[utf8]{inputenc}
\usepackage[T1]{fontenc}


\usepackage{amsmath}
\usepackage{amssymb} 
\usepackage{amsthm}  
\usepackage{dsfont}
\usepackage{mathrsfs}
\usepackage{mathtools}

\usepackage{geometry}

\usepackage{hyperref}        


\usepackage[french]{babel}

\usepackage[shortlabels]{enumitem}


\usepackage{fancyhdr}

\fancypagestyle{toc}{%
\fancyhf{}%
\fancyhead[L]{\rightmark}%
\fancyhead[R]{\thepage}%
}

\pagestyle{toc}

\newcommand{\indep}{\perp\!\!\! \perp}

\theoremstyle{plain}
\newtheorem{theorem}{Théorème}[section]
\newtheorem{coro}[theorem]{Corollaire}
\newtheorem{lemma}{Lemme}[section]
\newtheorem{prop}{Proposition}[section]

\theoremstyle{definition} 
\newtheorem{definition}{Définition}[section]
\newtheorem{axiom}{Axiome}[section]
\newtheorem{example}{Exemple}[subsection]
\newtheorem{exercice}{Exercice}[subsection]

\theoremstyle{plain}
\newtheorem{remarque}{Remarque}[subsection]

\newcommand{\N}{\mathbb{N}}
\newcommand{\Pa}{\mathscr{P}}


\begin{document}
\begin{titlepage}
	\newcommand{\HRule}{\rule{\linewidth}{0.5mm}}
	\center

	\HRule\\[0.4cm]

	\textsc{\Large Logique}\\[0.5cm]
	\textsc{\large Un ensemble compréhensible de notes de cours}\\[0.5cm]

	\HRule\\[1.5cm]

	{\large\textit{Auteur}}\\
	Yago \textsc{Iglesias}


	\vfill\vfill\vfill

	{\large\today}

	\vfill

\end{titlepage}

\tableofcontents

\section{Introduction}

Ce document est un recueil de notes de cours sur la logique niveau L3. Il est
basé sur les cours de Mme.~\textsc{Sylvy Anscombe} à Université Paris Cité, cependant toute
erreur ou inexactitude est de ma responsabilité.
Si bien \textsc{Yago IGLESIAS} est l'auteur de ce document, il n'est pas
le seul contributeur. Un remerciement particulier à \textsc{Gabin Dudillieu} pour sa
participation active à la rédaction de ce document. Tout futur contributeur
peut se retouver dans la section contributeurs du répertoire
\href{https://github.com/Yag000/logique-notes/graphs/contributors}{GitHub}.
\vspace{0.5cm}

Toute erreur signalée ou remarque est la bienvenue.
Sentez-vous libres de contribuer à ce document par le biais de \href{https://github.com/Yag000/logique-notes}{GitHub},
où vous pouvez trouver le code source de ce document et une version pdf à jour.
Si vous n'êtes pas familiers avec \textit{Git} ou \LaTeX, vous pouvez toujours me contacter
par \href{mailto: yago.iglesias.vazquez@gmail.com}{mail}.

\section{Comparaison des ensembles infinis}


\subsection{Axiome du choix}

On n'arrive pas à démontrer que tout ensemble infini contient un sous-ensemble dénombrable si on ne suppose pas AC.


\begin{definition}[Famille]
	$$ \left\{ (i, A_i) \mid i \in I \right\} $$

\end{definition}

\begin{axiom}[Formulation équiavalente de l'axiome du choix I]
	Pour toute famille non vide $(A_i)_{i \in I}$ il existe une fonction
	$f : I \to \bigcup{i \in I} A_i$ qui vérifie $F(i) \in A_i, \ \forall i$
\end{axiom}

\begin{axiom}[Formulation équiavalente de l'axiome du choix II]
	Si $(A_i)_{i \in I}$ est une famille non vide d'ensembles non vides,
	$$ \emptyset \neq \prod_{i \in I} A_i = \left\{ f: I \to \bigcup\limits_{i \in I} A_i  \mid f(i) \in A_i \right\} $$
\end{axiom}

\begin{theorem}
	Tout ensemble infini possède un sous ensemble dénombrable.
\end{theorem}

\begin{proof}
	Soit $E$ un ensemble infini $(\nexists n \in \N \mid E \cong N_{\leq n})$. Il suffit de trouver une fonction injective
	$\phi : \N \to E$. Soit $f: \Pa(E)\setminus\{\emptyset\} \to E$ du choix.\\
	On définit:
	$$ \Phi : \N \to \Pa(E) $$
	Donc $ \Phi (0) = E$.\\
	%TODO
\end{proof}

\begin{prop}
	Un ensemble est infini si et seulement si il est équipotent  une partie propre.
\end{prop}

\begin{proof}
	%TODO
\end{proof}




\section{Comparaison des ensembles infinis}


\subsection{Axiome du choix}

On n'arrive pas à démontrer que tout ensemble infini contient un sous-ensemble dénombrable si on ne suppose pas AC.


\begin{definition}[Famille]
	$$ \left\{ (i, A_i) \mid i \in I \right\} $$

\end{definition}

\begin{axiom}[Formulation équiavalente de l'axiome du choix I]
	Pour toute famille non vide $(A_i)_{i \in I}$ il existe une fonction
	$f : I \to \bigcup{i \in I} A_i$ qui vérifie $F(i) \in A_i, \ \forall i$
\end{axiom}

\begin{axiom}[Formulation équiavalente de l'axiome du choix II]
	Si $(A_i)_{i \in I}$ est une famille non vide d'ensembles non vides,
	$$ \emptyset \neq \prod_{i \in I} A_i = \left\{ f: I \to \bigcup\limits_{i \in I} A_i  \mid f(i) \in A_i \right\} $$
\end{axiom}

\begin{theorem}
	Tout ensemble infini possède un sous ensemble dénombrable.
\end{theorem}

\begin{proof}
	Soit $E$ un ensemble infini $(\nexists n \in \N \mid E \cong N_{\leq n})$. Il suffit de trouver une fonction injective
	$\phi : \N \to E$. Soit $f: \Pa(E)\setminus\{\emptyset\} \to E$ du choix.\\
	On définit:
	$$ \Phi : \N \to \Pa(E) $$
	Donc $ \Phi (0) = E$.\\
	%TODO
\end{proof}

\begin{prop}
	Un ensemble est infini si et seulement si il est équipotent  une partie propre.
\end{prop}

\begin{proof}
	%TODO
\end{proof}




\section{Comparaison des ensembles infinis}


\subsection{Axiome du choix}

On n'arrive pas à démontrer que tout ensemble infini contient un sous-ensemble dénombrable si on ne suppose pas AC.


\begin{definition}[Famille]
	$$ \left\{ (i, A_i) \mid i \in I \right\} $$

\end{definition}

\begin{axiom}[Formulation équiavalente de l'axiome du choix I]
	Pour toute famille non vide $(A_i)_{i \in I}$ il existe une fonction
	$f : I \to \bigcup{i \in I} A_i$ qui vérifie $F(i) \in A_i, \ \forall i$
\end{axiom}

\begin{axiom}[Formulation équiavalente de l'axiome du choix II]
	Si $(A_i)_{i \in I}$ est une famille non vide d'ensembles non vides,
	$$ \emptyset \neq \prod_{i \in I} A_i = \left\{ f: I \to \bigcup\limits_{i \in I} A_i  \mid f(i) \in A_i \right\} $$
\end{axiom}

\begin{theorem}
	Tout ensemble infini possède un sous ensemble dénombrable.
\end{theorem}

\begin{proof}
	Soit $E$ un ensemble infini $(\nexists n \in \N \mid E \cong N_{\leq n})$. Il suffit de trouver une fonction injective
	$\phi : \N \to E$. Soit $f: \Pa(E)\setminus\{\emptyset\} \to E$ du choix.\\
	On définit:
	$$ \Phi : \N \to \Pa(E) $$
	Donc $ \Phi (0) = E$.\\
	%TODO
\end{proof}

\begin{prop}
	Un ensemble est infini si et seulement si il est équipotent  une partie propre.
\end{prop}

\begin{proof}
	%TODO
\end{proof}




\section{Comparaison des ensembles infinis}


\subsection{Axiome du choix}

On n'arrive pas à démontrer que tout ensemble infini contient un sous-ensemble dénombrable si on ne suppose pas AC.


\begin{definition}[Famille]
	$$ \left\{ (i, A_i) \mid i \in I \right\} $$

\end{definition}

\begin{axiom}[Formulation équiavalente de l'axiome du choix I]
	Pour toute famille non vide $(A_i)_{i \in I}$ il existe une fonction
	$f : I \to \bigcup{i \in I} A_i$ qui vérifie $F(i) \in A_i, \ \forall i$
\end{axiom}

\begin{axiom}[Formulation équiavalente de l'axiome du choix II]
	Si $(A_i)_{i \in I}$ est une famille non vide d'ensembles non vides,
	$$ \emptyset \neq \prod_{i \in I} A_i = \left\{ f: I \to \bigcup\limits_{i \in I} A_i  \mid f(i) \in A_i \right\} $$
\end{axiom}

\begin{theorem}
	Tout ensemble infini possède un sous ensemble dénombrable.
\end{theorem}

\begin{proof}
	Soit $E$ un ensemble infini $(\nexists n \in \N \mid E \cong N_{\leq n})$. Il suffit de trouver une fonction injective
	$\phi : \N \to E$. Soit $f: \Pa(E)\setminus\{\emptyset\} \to E$ du choix.\\
	On définit:
	$$ \Phi : \N \to \Pa(E) $$
	Donc $ \Phi (0) = E$.\\
	%TODO
\end{proof}

\begin{prop}
	Un ensemble est infini si et seulement si il est équipotent  une partie propre.
\end{prop}

\begin{proof}
	%TODO
\end{proof}






\end{document}
